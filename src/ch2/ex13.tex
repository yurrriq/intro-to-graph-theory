\label{ch2ex:13}
\begin{marginfigure}
  \begin{align*}
    n + n &= \frac{n(n-1)}{2} \\
    2n &= \frac{n(n - 1)}{2} \\
    4n &= n(n-1) \\
    4n &= n^{2} - n \\
    0 &= n^{2} - 5n \\
    n &= \Set{0,5}
  \end{align*}
  \caption{Solving for $n$ in \hyperref[ch2ex:13]{Exercise~\ref{ch2ex:13}}}
\end{marginfigure}

If $C_{n} \cong \overline{C_{n}}$, then it must be the case that \\
$\abs*{E(C_{n})} + \abs*{E(\overline{C_{n}})} = \abs*{E(K_{n})}$, i.e.
$n + n = \frac{n(n-1)}{2}$.

\vskip 1em
Since $n=5$ is a solution,
$\abs*{E(C_{5})} + \abs*{E(\overline{C_{5}})} = \abs*{E(K_{5})}$, and thus
$C_{5} \cong \overline{C_{5}}$. \\
$\hfill\square$

\vskip 1em
A graph with no edges ($n=0$) has no cycles.
Therefore the only self-complementary cycle graph is $C_{5}$. \\
$\hfill\square$
